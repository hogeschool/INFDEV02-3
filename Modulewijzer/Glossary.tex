

\newglossaryentry{assignment}{name=Assignment, description={
    A large-sized task to be completed by a student in order to get a grade.
    Assignments are related to learning objectives.
    Students are expected to show their skill and understanding of those learning objectives within an assignment.}}

\newglossaryentry{exercise}{name=Exercise, description={
    A small- to medium-sized task to be completed by a student in order to gain experience, understanding and feedback. }}

\newglossaryentry{exam}{name=Theoretical exam, description={The summative assessment at the end of the course. Scoring $>=5.5$ is a strict condition for being rewarded credit points. (Dutch: tentamen).}}

\newglossaryentry{oral}{name=Oral check, description={The summative assessment at the end of the course, determining the final grade.}}

\newglossaryentry{competence}{name=Competence, description={A brief description of what students are expected to learn during the study programme (informatica).}}

\newglossaryentry{lo}{name=Learning objective, description={
    A brief description of what students are expected to learn during the course.
    In general, each course has between 1 and 10 learning objectives.}}

%\newglossaryentry{to}{name=Task objective, description={A brief description of what students are expected to learn during a task.}}

\newglossaryentry{feedback}{name=Feedback, description={
    "... feedback is conceptualized as information provided by an agent
    (e.g., teacher, peer, book, parent, self, experience) regarding aspects of one’s per- formance or understanding.
    A teacher or parent can provide corrective information, a peer can provide an alternative strategy,
    a book can provide information to clarify ideas, a parent can provide encouragement,
    and a learner can look up the answer to evaluate the correctness of a response.
    Feedback thus is a “consequence” of performance." \footnote{Hattie, J., \& Timperley, H. (2007). The Power of Feedback. Review of Educational Research, 77(1), 81–-112.}}
    }


\newglossaryentry{abs}{
      name=ABS
    , description={The \Gls{lo} stating that at the end of this course: the student \textbf{is able to use} and \textbf{create} interfaces and abstract classes.}
    , first={\textbf{is able to use} and \textbf{create} interfaces and abstract classes. \texttt{(ABS)}}
    , symbol=\texttt{ABS}
    }
\newglossaryentry{learn}{
      name=LEARN
    , description={The \Gls{lo} stating that at the end of this course: the student \textbf{has developed skills} to adopt a new programming language with little support.}
    , first={\textbf{has developed skills} to adopt a new programming language with little support. \texttt{(LEARN)}}
    , symbol=\texttt{LEARN}
    }
\newglossaryentry{enc}{
      name=ENC
    , description={The \Gls{lo} stating that at the end of this course: the student \textbf{is able to apply} the concepts of data encapsulation, inheritance, and polymorphism to software.}
    , first={\textbf{is able to apply} the concepts of data encapsulation, inheritance, and polymorphism to software. \texttt{(ENC)}}
    , symbol=\texttt{ENC}
    }
\newglossaryentry{type}{
      name=TYPE
    , description={The \Gls{lo} stating that at the end of this course: the student \textbf{can apply} the concepts of data types.}
    , first={\textbf{can apply} the concepts of data types. \texttt{(TYPE)}}
    , symbol=\texttt{TYPE}
    }
\newglossaryentry{bhf}{
      name=BHF
    , description={The \Gls{lo} stating that at the end of this course: the student \textbf{understands} basic human factors.}
    , first={\textbf{understands} basic human factors. \texttt{(BHF)}}
    , symbol=\texttt{BHF}
    }
