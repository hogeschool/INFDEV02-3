\documentclass[titlepage, a4paper, openbib, 10pt]{article}

%#####################################
%Usepackages en installingen
\usepackage[top=1in, bottom=1in, left=1in, right=1in]{geometry}
\usepackage[pdftex]{graphicx}
\usepackage{fancyhdr}
\usepackage{sectionbox}
\usepackage[english]{babel}
\usepackage{chngcntr}
\usepackage{cite}
\usepackage{url}
\usepackage{makeidx}
\usepackage{paralist}
\usepackage{enumitem}
\usepackage{tocloft}
\usepackage{listliketab}	
\usepackage[table]{xcolor}
\usepackage{tabularx}
\usepackage{epsfig}
\usepackage{pdflscape}
\usepackage{pdfpages}
\usepackage{float}
\usepackage{multirow} 
\usepackage{rotating}
\usepackage[utf8]{inputenc}
\usepackage{color}
\usepackage{fp}
\usepackage[hidelinks]{hyperref}
\hypersetup{
    colorlinks=false,
    linkcolor=black,
    filecolor=black,
    urlcolor=black,
}
%\usepackage{draftwatermark}
%\SetWatermarkText{\textsc{Draft}}
%\SetWatermarkScale{5}
\newcommand{\red}[1]{
\textcolor{red}{#1}
}
\usepackage{listings}
\lstset{language=C,
basicstyle=\ttfamily\footnotesize,
frame=shadowbox,
mathescape=true,
showstringspaces=false,
showspaces=false,
breaklines=true}

%#####################################
%Glossary
%\usepackage{glossaries}
\usepackage[toc,nonumberlist,nopostdot]{glossaries}
\makeglossaries


\newglossaryentry{assignment}{name=Assignment, description={
    A large-sized task to be completed by a student in order to get a grade.
    Assignments are related to learning objectives.
    Students are expected to show their skill and understanding of those learning objectives within an assignment.}}

\newglossaryentry{exercise}{name=Exercise, description={
    A small- to medium-sized task to be completed by a student in order to gain experience, understanding and feedback. }}

\newglossaryentry{exam}{name=Theoretical exam, description={The summative assessment at the end of the course. Scoring $>=5.5$ is a strict condition for being rewarded credit points. (Dutch: tentamen).}}

\newglossaryentry{oral}{name=Oral check, description={The summative assessment at the end of the course, determining the final grade.}}

\newglossaryentry{competence}{name=Competence, description={A brief description of what students are expected to learn during the study programme (informatica).}}

\newglossaryentry{lo}{name=Learning objective, description={
    A brief description of what students are expected to learn during the course.
    In general, each course has between 1 and 10 learning objectives.}}

%\newglossaryentry{to}{name=Task objective, description={A brief description of what students are expected to learn during a task.}}

\newglossaryentry{feedback}{name=Feedback, description={
    "... feedback is conceptualized as information provided by an agent
    (e.g., teacher, peer, book, parent, self, experience) regarding aspects of one’s per- formance or understanding.
    A teacher or parent can provide corrective information, a peer can provide an alternative strategy,
    a book can provide information to clarify ideas, a parent can provide encouragement,
    and a learner can look up the answer to evaluate the correctness of a response.
    Feedback thus is a “consequence” of performance." \footnote{Hattie, J., \& Timperley, H. (2007). The Power of Feedback. Review of Educational Research, 77(1), 81–-112.}}
    }


\newglossaryentry{abs}{
      name=ABS
    , description={The \Gls{lo} stating that at the end of this course: the student \textbf{is able to use} and \textbf{create} interfaces and abstract classes.}
    , first={\textbf{is able to use} and \textbf{create} interfaces and abstract classes. \texttt{(ABS)}}
    , symbol=\texttt{ABS}
    }
\newglossaryentry{learn}{
      name=LEARN
    , description={The \Gls{lo} stating that at the end of this course: the student \textbf{has developed skills} to adopt a new programming language with little support.}
    , first={\textbf{has developed skills} to adopt a new programming language with little support. \texttt{(LEARN)}}
    , symbol=\texttt{LEARN}
    }
\newglossaryentry{enc}{
      name=ENC
    , description={The \Gls{lo} stating that at the end of this course: the student \textbf{is able to apply} the concepts of data encapsulation, inheritance, and polymorphism to software.}
    , first={\textbf{is able to apply} the concepts of data encapsulation, inheritance, and polymorphism to software. \texttt{(ENC)}}
    , symbol=\texttt{ENC}
    }
\newglossaryentry{type}{
      name=TYPE
    , description={The \Gls{lo} stating that at the end of this course: the student \textbf{can apply} the concepts of data types.}
    , first={\textbf{can apply} the concepts of data types. \texttt{(TYPE)}}
    , symbol=\texttt{TYPE}
    }
\newglossaryentry{bhf}{
      name=BHF
    , description={The \Gls{lo} stating that at the end of this course: the student \textbf{understands} basic human factors.}
    , first={\textbf{understands} basic human factors. \texttt{(BHF)}}
    , symbol=\texttt{BHF}
    }

%\loadglsentries{Glossary}


%\usepackage{showframe} %tmp
%#####################################
%Nieuwe commando's
\newcommand{\HRule}{\rule{\linewidth}{1pt}}
\newcommand{\organisatie}{\uppercase{Hogeschool Rotterdam / CMI}}
\newcommand{\modulenaam}{Development 3}
\newcommand{\modulecode}{\uppercase{INFDEV02-3}}
\newcommand{\studiejaar}{\uppercase{2015-2016}}
\newcommand{\stdPunten}{4}
\renewcommand{\author}{Youri Tjang \& Giuseppe Maggiore}

\definecolor{lichtGrijs}{RGB}{169,169,169}



%#####################################
%Index en styling
\setlength{\cftbeforesecskip}{10pt}
\setlength\parindent{0pt}
\makeindex
\graphicspath{{../Img/}}
\counterwithin{figure}{subsection}
\pagestyle{fancy}
\setcounter{secnumdepth}{5}
\setcounter{tocdepth}{5}

%#####################################
%     Alles voor header/footer
\fancyhf[HL]{\nouppercase{\textit{\leftmark}}}
\setlength{\headheight}{36pt}
\lhead{\uppercase{\footnotesize Course description}}
\chead{\footnotesize \organisatie}
\rhead{\includegraphics[width=0.09\textwidth]{logo}}

\lfoot{\scriptsize \modulenaam}
\cfoot{\scriptsize \today}
\rfoot{\small \thepage}

\renewcommand{\headrulewidth}{0.4pt}
\renewcommand{\footrulewidth}{0.4pt}
%#####################################

\begin{document}


%#####################################
%Titlepage
\begin{titlepage}
\thispagestyle{fancy}
\input{Voorblad}
\end{titlepage}

%####### Contentpagina ########
%\renewcommand{\baselinestretch}{1.5}\normalsize
%\tableofcontents
%\newpage
%\listoffigures
%\newpage
%\listoftables
%\newpage

%########### Inhoud ###########

\shadowsectionbox
\section*{Modulebeschrijving}
\begin{tabularx}{\textwidth}{|>{\columncolor{lichtGrijs}} p{.26\textwidth}|X|}
	\hline
	\textbf{Module name:} & \modulenaam\\

	\hline
	\textbf{Module code: }& \modulecode\\
	\hline
	\textbf{Study points \newline and hours of effort:} & This module gives \stdPunten{}  ects, in correspondence with \FPeval{\result}{clip(\stdPunten*28)}\result{} hours:
	\begin{itemize}
		\item 2 X 3 x 6 hours of combined lecture and practical
		\item the rest is self-study
	\end{itemize} \\
	\hline
	\textbf{Examination:} & Written examination and practicums (with oral check) \\
	\hline
	\textbf{Course structure:} & Lectures, self-study, and practicums \\
	\hline
	\textbf{Prerequisite knowledge:} & INFDEV02-1 and INFDEV02-2. \\
	\hline
	\textbf{Learning tools:}  &
		\begin{itemize}
			\item Book: Think Java; author A. B. Downey (\href{http://www.greenteapress.com/thinkjava/}{www.greenteapress.com/thinkjava})
			\item Book: Head First Java (2nd ed.); authors K. Sierra, \& B. Bates. (2005).
			\item Presentations: found on N@tschool and on the GitHub repository \href{https://github.com/hogeschool/INFDEV02-3}{github.com/hogeschool/INFDEV02-3}
			\item Assignments, to be done at home and during practical part of the lectures (pdf): found on N@tschool and on the GitHub repository \href{https://github.com/hogeschool/INFDEV02-3}{github.com/hogeschool/INFDEV02-3}
		\end{itemize} \\
	\hline
	\textbf{Connected to competences:} & realiseren en ontwerpen \\
	\hline
	\textbf{Learning objectives:} &
		At the end of the course, the student:
			\begin{itemize}
                \item \textbf{is able to use} and \textbf{create} interfaces and abstract classes. \texttt{(ABS)}
                \item \textbf{has developed skills} to adopt a new programming language with little support. \texttt{(LEARN)}
                \item \textbf{is able to apply} the concepts of data encapsulation, inheritance, and polymorphism to software. \texttt{(ENC)}
                \item \textbf{can apply} the concepts of data types. \texttt{(TYPE)}
                \item \textbf{understands} basic human factors. \texttt{(BHF)}
			\end{itemize} \\
	\hline
%\end{tabularx}
%\newpage
%
%\begin{tabularx}{\textwidth}{|>{\columncolor{lichtGrijs}} p{.26\textwidth}|X|}
%	\hline
%	\textbf{Content:}&
%	\begin{itemize}
%		\item _
%	\end{itemize} \\
%	\hline
	\textbf{Course owners:} & \author\\
	\hline
	\textbf{Date:} & \today \\
	\hline
\end{tabularx}
%\newpage


\newpage
\include{AlgemeneOmschrijving}
%\newpage
\section{Course program}
The course is structured into four(4) chapters. The four chapters take place during the six weeks of the course.

\subsection{Chapter 1 - statically typed programming languages}
\paragraph*{Topics}
\begin{itemize}
	\item What are types?
	\item (\textbf{Advanced}) Typing and semantic rules: how do we read them?
	\item Introduction to Java and C\# (\textbf{advanced}) with type rules and semantics
	\begin{itemize}
		\item Classes
		\item Fields/attributes
		\item Constructor(s), methods, and static methods
		\item Statements, expressions, and primitive types
		\item Arrays
		\item (\textbf{Advanced}) Lambda's
	\end{itemize}
\end{itemize}

\subsection{Chapter 2 - reuse through polymorphism}
\paragraph*{Topics}			
\begin{itemize}
	\item What is code reuse?
	\item Interfaces, abstract classes and implementation
	\item Implicit vs explicit conversion
	\item (\textbf{Advanced}) Implicit and explicit conversion type rules
	\item Runtime type testing
\end{itemize}


%\subsection{Chapter 3 - reuse through generics}
%\paragraph*{Topics}			
%\begin{itemize}
%	\item Using generic parameters
%	\item (\textbf{Advanced}) Using covariance and contravariance in the presence of generic parameters
%	\item (\textbf{Advanced}) Designing interfaces and implementation in the presence of generic parameters
%\end{itemize}


\subsection{Chapter 3 - architectural considerations}
\paragraph*{Topics}			
\begin{itemize}
	\item Encapsulation
	\item Input controllers
	\item State machines
\end{itemize}

\subsection{Chapter 4 - yet more architectural considerations}
\paragraph*{Topics}			
\begin{itemize}
	\item (\textbf{Advanced}) Composition versus inheritance
	\item (\textbf{Advanced}) Entity/component model
\end{itemize}

\newpage
\section{Assessment}
The course is tested by means of a written and a practicum exam:
Moreover, you have to deliver (on N@tschool) a series of \glspl{assignment} which will not be graded but are mandatory. The \gls{oral} is based on the \glspl{assignment}, wereas the written exam is based on the theory introduced in the course. The final grade is determined as follows: \\

\texttt{if \gls{exam}-grade $ >= 5.5 $ then return \gls{oral}-grade else return 0}

\paragraph*{Motivation for grade}
A professional software developer is required to be able to program code which is, at the very least, \textit{correct}.

In order to produce correct code, we expect students to show:
\begin{inparaenum}[\itshape i\upshape)]
\item a foundation of knowledge about how a programming language actually works in connection with a simplified concrete model of a computer;
\item fluency when actually writing the code.
\end{inparaenum}

The quality of the programmer is ultimately determined by his actual code-writing skills, therefore the written exam will contain require you to write code, this ensures that each student is able to show that his work is his own and that he has adequate understanding of its mechanisms.



\subsection{Theoretical examination \modulecode}
The general shape of a \gls{exam} for \texttt{\modulecode} is made up of a short series of highly structured questions.
In each exam the content of the questions will change, but the structure of the questions will remain the same.
For the structure (and an example) of the theoretical exam, see the appendix.


\subsection{Practical examination \modulecode}
There are 2 \glspl{assignment} which are mandatory, and formatively assessed for \gls{feedback}.

\begin{itemize}
  \item All assignments are to be uploaded to N@tschool in the required space (Inlevermap);
  \item Each assignment is designed to assess the students knowledge related to one or more \glspl{lo}.
          If the teacher is unable te assess the students' ability related to the appropriate \gls{lo} based on his work, then no points will be awarded for that part.
  \item \textit{The teachers still reserve the right to check the practicums handed in by each student, and to use it for further evaluation.}
  \item The university rules on fraude and plagiarism (Hogeschoolgids art. 11.10 -- 11.12) also apply to code;
\end{itemize}

\newpage
\section*{Structure of exam \modulecode}
The general shape of a theoretical exam for \texttt{DEV 3} is made up of only two, highly structured open questions.

\paragraph{Question 1: } \ \\

\textbf{General shape of the question:} \textit{Given the following class definitions, and a piece of code that uses them, fill in the stack, heap, and PC with all steps taken by the program at runtime.}

\textbf{Concrete example of question:}

\lstset{numbers=left,basicstyle=\ttfamily\small}\lstset{language=[Sharp]C}
\begin{lstlisting}
interface ICounter {
  void Incr(int diff);
}
class Counter : ICounter {
  private int cnt;
  public Counter() {
    this.cnt = 0;
  }
  public void Incr(int diff) {
    this.cnt = (this.cnt + diff);
  }
}
ICounter c = new Counter();
c.Incr(5);
\end{lstlisting}

\textbf{Concrete example of answer:}

{
\small
\begin{enumerate}
\item Stack: 
\begin{tabular}{ |c| }
		\hline
		PC \\
		\hline
		1 \\
		\hline
		
\end{tabular}
	\\

\item Stack: \begin{tabular}{ |c| }
		\hline
		PC \\
		\hline
		13 \\
		\hline
		
	\end{tabular}

Heap: \begin{tabular}{ |c| }
		\hline
		1 \\
		\hline
		\begin{tabular}{c} cnt= \end{tabular} \\
		\hline
		
	\end{tabular}\\

\item Stack: \begin{tabular}{ |c|c|>{\columncolor[gray]{0.8}}c|c|c|c| }
		\hline
		PC & ... & & PC & ret & this \\
		\hline
		13 & ...
		& & 7 & null & ref 1 \\
		\hline
		
	\end{tabular}

Heap: 
\begin{tabular}{ |c| }
		\hline
		1 \\
		\hline
		\begin{tabular}{c} cnt= \end{tabular} \\
		\hline
		
	\end{tabular}\\

\item Stack: \begin{tabular}{ |c|c|>{\columncolor[gray]{0.8}}c|c|c| }
		\hline
		PC & ... & & PC & ret \\
		\hline
		13 & ...
		& & 7 & ref 1 \\
		\hline
		
	\end{tabular}
	\\Heap: \begin{tabular}{ |c| }
		\hline
		1 \\
		\hline
		\begin{tabular}{c} cnt=0 \end{tabular} \\
		\hline
		
	\end{tabular}\\

\item Stack: \begin{tabular}{ |c|c| }
		\hline
		PC & c \\
		\hline
		14 & ref 1 \\
		\hline
		
	\end{tabular}
	\\Heap: \begin{tabular}{ |c| }
		\hline
		1 \\
		\hline
		\begin{tabular}{c} cnt=0 \end{tabular} \\
		\hline
		
	\end{tabular}\\

\item Stack: \begin{tabular}{ |c|c|>{\columncolor[gray]{0.8}}c|c|c|c|c| }
		\hline
		PC & ... & & PC & ret & diff & this \\
		\hline
		14 & ...
		& & 10 & null & 5 & ref 1 \\
		\hline
		
	\end{tabular}
	\\Heap: \begin{tabular}{ |c| }
		\hline
		1 \\
		\hline
		\begin{tabular}{c} cnt=0 \end{tabular} \\
		\hline
		
	\end{tabular}\\

\item Stack: \begin{tabular}{ |c|c|>{\columncolor[gray]{0.8}}c|c|c| }
		\hline
		PC & ... & & PC & ret \\
		\hline
		14 & ...
		& & 10 & ref 1 \\
		\hline
		
	\end{tabular}
	\\Heap: \begin{tabular}{ |c| }
		\hline
		1 \\
		\hline
		\begin{tabular}{c} cnt=5 \end{tabular} \\
		\hline
		
	\end{tabular}\\

\item Stack: \begin{tabular}{ |c|c| }
		\hline
		PC & c \\
		\hline
		15 & ref 1 \\
		\hline
		
	\end{tabular}
	\\Heap: \begin{tabular}{ |c| }
		\hline
		1 \\
		\hline
		\begin{tabular}{c} cnt=5 \end{tabular} \\
		\hline
		
	\end{tabular}

\end{enumerate}
}

\textbf{Points:} \textit{4 (50\% of total).}

\textbf{Grading:} \textit{Full points for more than 90\% of correct names and values. Three points if at least all names are correctly placed on the stack with at least half the values correct. Half points for more than 40\% of correct names and values. Zero points otherwise.}

\textbf{Associated learning objective:} \glsfirst{abs}

\ \\ 

\paragraph{Question 2: } \ \\

\textbf{General shape of question:} \textit{Given the following class definitions, and a piece of code that uses them, fill in the declarations, class definitions, and PC with all steps taken by the compiler while type checking.}

\textbf{Concrete example of question:} 

\lstset{numbers=left,basicstyle=\ttfamily\small}\lstset{language=[Sharp]C}
\begin{lstlisting}
interface ICounter {
  void Incr(int diff);
}
class Counter : ICounter {
  private int cnt;
  public Counter() {
    this.cnt = 0;
  }
  public void Incr(int diff) {
    this.cnt = (this.cnt + diff);
  }
}
ICounter c = new Counter();
c.Incr(5);
\end{lstlisting}

\textbf{Concrete example of answer:} \textit{}

{
	\small
	\begin{enumerate}
\item Declarations: \begin{tabular}{ |c| }
	\hline
	PC \\
	\hline
	1 \\
	\hline
	
\end{tabular}

\item Declarations: \begin{tabular}{ |c| }
		\hline
		PC \\
		\hline
		4 \\
		\hline
		
	\end{tabular}
	\\Classes: \begin{tabular}{ |c|c|c|c|c|c|c|c| }
		\hline
		ICounter \\
		\hline
		\begin{tabular}{c} Incr=(ICounter$\times$int) $\rightarrow$ void \end{tabular} \\
		\hline
		
	\end{tabular}

\item Declarations: \begin{tabular}{ |c|c| }
		\hline
		PC & this \\
		\hline
		7 & Counter \\
		\hline
		
	\end{tabular}
	\\Classes: \begin{tabular}{ |c|c|c|c|c|c|c|c|c|c|c|c|c|c|c|c|c|c| }
		\hline
		Counter & ICounter \\
		\hline
		\begin{tabular}{c} Counter=Counter $\rightarrow$ Counter \\Incr=(Counter$\times$int) $\rightarrow$ void \\cnt=int \end{tabular} & \begin{tabular}{c} Incr=(ICounter$\times$int)$\rightarrow$ void \end{tabular} \\
		\hline
		
	\end{tabular}

\item Declarations: \begin{tabular}{ |c|c|c| }
		\hline
		PC & diff & this \\
		\hline
		9 & int & Counter \\
		\hline
		
	\end{tabular}
	\\Classes: \begin{tabular}{ |c|c|c|c|c|c|c|c|c|c|c|c|c|c|c|c|c|c| }
		\hline
		Counter & ICounter \\
		\hline
		\begin{tabular}{c} Counter=Counter $\rightarrow$ Counter \\Incr=(Counter$\times$int) $\rightarrow$ void \\cnt=int \end{tabular} & \begin{tabular}{c} Incr=(ICounter$\times$int) $\rightarrow$ void \end{tabular} \\
		\hline
		
	\end{tabular}

\item Declarations: \begin{tabular}{ |c| }
		\hline
		PC \\
		\hline
		13 \\
		\hline
		
	\end{tabular}
	\\Classes: \begin{tabular}{ |c|c|c|c|c|c|c|c|c|c|c|c|c|c|c|c|c|c| }
		\hline
		Counter & ICounter \\
		\hline
		\begin{tabular}{c} Counter=Counter $\rightarrow$ Counter \\Incr=(Counter$\times$int) $\rightarrow$ void \\cnt=int \end{tabular} & \begin{tabular}{c} Incr=(ICounter$\times$int) $\rightarrow$ void \end{tabular} \\
		\hline
		
	\end{tabular}

\item Declarations: \begin{tabular}{ |c|c| }
		\hline
		PC & c \\
		\hline
		14 & ICounter \\
		\hline
		
	\end{tabular}
	\\Classes: \begin{tabular}{ |c|c|c|c|c|c|c|c|c|c|c|c|c|c|c|c|c|c| }
		\hline
		Counter & ICounter \\
		\hline
		\begin{tabular}{c} Counter=Counter $\rightarrow$ Counter \\Incr=(Counter$\times$int) $\rightarrow$ void \\cnt=int \end{tabular} & \begin{tabular}{c} Incr=(ICounter$\times$int) $\rightarrow$ void \end{tabular} \\
		\hline
		
	\end{tabular}

\item Declarations: \begin{tabular}{ |c|>{\columncolor[gray]{0.8}}c|c|c|c|c| }
		\hline
		c & & PC & ret & $arg_1$ & this \\
		\hline
		ICounter & & 14 & null & int & Counter \\
		\hline
		
	\end{tabular}
	\\Classes: \begin{tabular}{ |c|c|c|c|c|c|c|c|c|c|c|c|c|c|c|c|c|c| }
		\hline
		Counter & ICounter \\
		\hline
		\begin{tabular}{c} Counter=Counter $\rightarrow$ Counter \\Incr=(Counter$\times$int) $\rightarrow$ void \\cnt=int \end{tabular} & \begin{tabular}{c} Incr=(ICounter$\times$int) $\rightarrow$ void \end{tabular} \\
		\hline
		
	\end{tabular}

\item Declarations: \begin{tabular}{ |c|>{\columncolor[gray]{0.8}}c|c|c|c|c| }
		\hline
		c & & PC & ret & $arg_1$ & this \\
		\hline
		ICounter & & 14 & void & int & Counter \\
		\hline
		
	\end{tabular}
	\\Classes: \begin{tabular}{ |c|c|c|c|c|c|c|c|c|c|c|c|c|c|c|c|c|c| }
		\hline
		Counter & ICounter \\
		\hline
		\begin{tabular}{c} Counter=Counter $\rightarrow$ Counter \\Incr=(Counter$\times$int) $\rightarrow$ void \\cnt=int \end{tabular} & \begin{tabular}{c} Incr=(ICounter$\times$int) $\rightarrow$ void \end{tabular} \\
		\hline
		
	\end{tabular}

\item Declarations: \begin{tabular}{ |c|c| }
		\hline
		PC & c \\
		\hline
		15 & ICounter \\
		\hline
		
	\end{tabular}
	\\Classes: \begin{tabular}{ |c|c|c|c|c|c|c|c|c|c|c|c|c|c|c|c|c|c| }
		\hline
		Counter & ICounter \\
		\hline
		\begin{tabular}{c} Counter=Counter $\rightarrow$ Counter \\Incr=(Counter$\times$int) $\rightarrow$ void \\cnt=int \end{tabular} & \begin{tabular}{c} Incr=(ICounter$\times$int) $\rightarrow$ void \end{tabular} \\
		\hline
		
	\end{tabular}
	
	\end{enumerate}
}

\textbf{Points:} \textit{4 (50\% of total).}

\textbf{Grading:} \textit{Full points for more than 90\% of correct names and types. Three points if at least all names are correctly placed on the declarations and classes, with at least half the types correct. Half points for more than 40\% of correct names and types. Zero points otherwise.}

\textbf{Associated learning objective:} \glsfirst{type}

\ \\

\newpage

\newpage
%\bibliographystyle{plain}
%\bibliography{references}
\glossarystyle{altlist}
\printglossaries
\newpage

\section*{Bijlage 1: Toetsmatrijs}
	\begin{tabular}{|p{2cm}|p{4cm}|}
		\hline
		Learning goals & Dublin descriptors \\
		\hline
        \texttt{ABS} & 1, 2, 4\\
        \hline
        \texttt{LEARN}& 1, 4, 5\\
        \hline
        \texttt{ENC} & 1, 2, 4\\
        \hline
        \texttt{TYPE} & 1, 2, 4\\
        \hline
        \texttt{BHF} & 1, 2, 4\\
        \hline
	\end{tabular}
	
	\vspace{1cm}

	Dublin-descriptors:
	\begin{enumerate}
		\item Knowledge and understanding
		\item Applying knowledge and understanding
		\item Making judgments
		\item Communication
		\item Learning skills
	\end{enumerate}


%\newpage
%\input{tex/Bijlage2}
%\newpage
%\input{tex/Bijlage3}
\printindex


\end{document}

