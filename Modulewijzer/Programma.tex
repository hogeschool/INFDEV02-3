\section{Course program}
The course is structured into six lectures. The six lectures take place during the six weeks of the course, but are not necessarily in a one-to-one correspondance with the course weeks. For example, lectures one and two are fairly short and can take place during a single week.

\subsection{Chapter 1 - statically typed programming languages}
\paragraph*{Topics}
\begin{itemize}
	\item What are types?
	\item (\textbf{Advanced}) Typing and semantic rules: how do we read them?
	\item Introduction to Java and C\# (\textbf{advanced}) with type rules and semantics
	\begin{itemize}
		\item Classes
		\item Fields/attributes
		\item Constructor(s), methods, and static methods
		\item Statements, expressions, and primitive types
		\item Arrays
		\item (\textbf{Advanced}) Lambda's
	\end{itemize}
\end{itemize}


\subsection{Chapter 2 - reuse through polymorphism}

\paragraph*{Topics}			
\begin{itemize}
	\item What is code reuse?
	\item Interfaces and implementation
	\item Implicit vs explicit conversion
	\item (\textbf{Advanced}) Implicit and explicit conversion type rules
	\item Runtime type testing
\end{itemize}

\subsection{Chapter 3 - reuse through generics}
\paragraph*{Topics}			
\begin{itemize}
	\item (\textbf{Advanced}) Generic parameters
	\item (\textbf{Advanced}) Interfaces and implementation in the presence of generic parameters
	\item (\textbf{Advanced}) Covariance and contravariance in the presence of generic parameters
\end{itemize}



\subsection{Chapter 4 - architectural considerations}

\paragraph*{Topics}			
\begin{itemize}
	\item Encapsulation
	\item Abstract classes: between interfaces and implementation
	\item Inheritance of classes and abstract classes
\end{itemize}

\subsection{Chapter 5 - yet more architectural considerations}
\paragraph*{Topics}			
\begin{itemize}
	\item (\textbf{Advanced}) Composition versus inheritance
	\item (\textbf{Advanced}) Entity/component model
\end{itemize}
	
\end{enumerate}