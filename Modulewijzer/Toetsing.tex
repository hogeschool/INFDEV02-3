\section{Assessment}
The course is tested with two exams:
a series of practical assignments,
and a written exam. The final grade is determined as follows: \\

\texttt{if $practicumCheckOK$ then return writtenExamGrade else return 0}

The written exam will include questions about the practical assignments as well as theoretical topics.


\paragraph*{Motivation for grade}
A professional software developer is required to be able to program code which is, at the very least, \textit{correct}.

In order to produce correct code, we expect students to show:
\begin{inparaenum}[\itshape i\upshape)]
\item a foundation of knowledge about how a programming language actually works in connection with a simplified concrete model of a computer;
\item fluency when actually writing the code.
\end{inparaenum}

The quality of the programmer is ultimately determined by his actual code-writing skills, therefore the written exam will contain require you to write code, this ensures that each student is able to show that his work is his own and that he has adequate understanding of its mechanisms.



\subsection{Theoretical examination \modulecode}
The general shape of a theoretical exam for \texttt{\modulecode} is made up of a series of highly structured open questions.
In each exam the content of the questions will change, but the structure of the questions will remain the same.
For the structure (and an example) of the theoretical exam, see the appendix.


\subsection{Practical examination \modulecode}
Each week there is a mandatory assignment. The assignments of week 4, 5 and 6 will be graded.
Each assignment is due the following week.
The sum of the grades will be the $practicumGrade$.
If the course is over and $practicumGrade$ is lower than $5,5$ then you can retry (herkansing) the practicum with one assignment which will test all learning objectives and will replace the whole $practicumGrade$.
If the $practicumGrade$ is $5,5$ or above then $practicumCheckOK$.
The following rules apply to the assignment:
\begin{itemize}
  \item All assignments are to be uploaded to N@tschool of Classroom in the required space (Inlevermap or assignment);
  \item Each assignment is designed to assess the students knowledge related to one or more learning objectives.
          The relevant learning objective will be stated above the assignment.
          If the teacher is unable te assess the student's ability based on his work, then no points will be awarded for that part.
  \item The university rules on fraude and plagiarism (Hogeschoolgids art. 11.10 -- 11.12) also apply to code;
\end{itemize}


\textit{The teachers still reserve the right to check the practicums handed in by each student, and to use it for further evaluation.}

